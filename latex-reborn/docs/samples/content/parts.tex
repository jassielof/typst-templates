% Muestras de \part{} con títulos largos y cortos, con y sin contenido.

\part{Introducción}
% Sin contenido

\part[Marco teórico]{Marco teórico y revisión bibliográfica exhaustiva sobre los fundamentos conceptuales, metodológicos y tecnológicos del proyecto, incluyendo antecedentes, definiciones, taxonomías, patrones de diseño, estándares aplicables y trabajos relacionados en el ámbito local e internacional}
Este es un texto de ejemplo para ilustrar contenido dentro de una parte. Aquí se presenta un panorama general y se destacan los objetivos que guían el resto del documento.

\part[Diseño]{Diseño de la solución, arquitectura de software, decisiones tecnológicas, compromisos y trade-offs, gestión de riesgos, lineamientos de calidad y plan de validación y verificación de la implementación propuesta}
Contenido de ejemplo que describe la estructura de la solución, sus componentes principales y las interacciones entre subsistemas. También se indican criterios de aceptación y métricas de calidad.

\part{Resultados}
Breve contenido de ejemplo con un resumen de hallazgos, métricas comparativas y observaciones clave derivadas de las pruebas realizadas.

\part[Conclusiones]{Conclusiones y trabajo futuro}
% Sin contenido

\part[Apéndices]{Apéndices y material complementario para consulta, incluyendo manuales de usuario, configuraciones de despliegue, scripts de migración, casos de prueba y documentación técnica detallada}
% Sin contenido
