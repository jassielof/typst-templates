\chapter{Ejemplos exhaustivos de figuras, tablas, listados y ecuaciones}

Este capítulo muestra ejemplos completos de uso de figuras, tablas, listados de código y ecuaciones, con referencias cruzadas.

\section{Figuras (marcadores de imagen)}
En la figura \ref{fig:placeholder1} se muestra un marcador simple. En la figura \ref{fig:placeholder2} se usa un recuadro con texto centrado. La figura \ref{fig:placeholder-side-by-side} muestra dos marcadores en paralelo.

\begin{figure}[htbp]
  \centering
  % Rectángulo vacío de 0.9\textwidth por 6cm
  \fbox{\rule{0pt}{6cm}\rule{0.9\textwidth}{0pt}}
  \caption{Marcador de imagen vacío (proporción libre)}
  \label{fig:placeholder1}
\end{figure}

\begin{figure}[htbp]
  \centering
  % Recuadro con texto centrado, útil como “placeholder” con título
  \fbox{%
    \begin{minipage}[c][6cm][c]{0.9\textwidth}
      \centering
      Marcador de imagen 16:9 \par
      (Reemplace por una imagen real cuando esté disponible)
    \end{minipage}%
  }
  \caption{Marcador de imagen con texto centrado}
  \label{fig:placeholder2}
\end{figure}

\begin{figure}[htbp]
  \centering
  \begin{minipage}{0.48\textwidth}
    \centering
    \fbox{\rule{0pt}{4cm}\rule{0.95\textwidth}{0pt}}
    \par (a) Ejemplo A
  \end{minipage}\hfill
  \begin{minipage}{0.48\textwidth}
    \centering
    \fbox{\rule{0pt}{4cm}\rule{0.95\textwidth}{0pt}}
    \par (b) Ejemplo B
  \end{minipage}
  \caption{Dos marcadores de imagen en paralelo}
  \label{fig:placeholder-side-by-side}
\end{figure}

\section{Tablas}
La tabla \ref{tab:basica} muestra una tabla básica. La tabla \ref{tab:agrupada} incluye cabeceras agrupadas.

\begin{table}[htbp]
  \centering
  \caption{Tabla básica con tres columnas}
  \label{tab:basica}
  \begin{tabular}{lcr}
    \hline
    Columna A & Columna B & Columna C \\
    \hline
    Alfa  & 10 & Sí \\
    Beta  & 20 & No \\
    Gamma & 30 & Sí \\
    \hline
  \end{tabular}
\end{table}

\begin{table}[htbp]
  \centering
  \caption{Tabla con cabecera agrupada}
  \label{tab:agrupada}
  \begin{tabular}{lcc}
    \hline
    & \multicolumn{2}{c}{Medidas} \\
    \cline{2-3}
    Ítem   & Ancho & Alto \\
    \hline
    A      & 12    & 5 \\
    B      & 7     & 3 \\
    C      & 9     & 8 \\
    \hline
  \end{tabular}
\end{table}

\section{Listados de código}
A continuación, se muestran listados de código incluidos en un entorno flotante para permitir pies de figura y etiquetas.

\begin{figure}[htbp]
  \centering
  \begin{minipage}{0.95\linewidth}
\begin{verbatim}
# Python: función Fibonacci con memoización simple
def fibonacci(n, cache=None):
    if cache is None:
        cache = {}
    if n < 2:
        return n
    if n in cache:
        return cache[n]
    cache[n] = fibonacci(n-1, cache) + fibonacci(n-2, cache)
    return cache[n]

print([fibonacci(i) for i in range(10)])
\end{verbatim}
  \end{minipage}
  \caption{Listado: función en Python}
  \label{lst:python}
\end{figure}

\begin{figure}[htbp]
  \centering
  \begin{minipage}{0.95\linewidth}
\begin{verbatim}
{
  "service": "api",
  "version": 1,
  "features": ["auth", "cache", "metrics"],
  "limits": { "rate_per_minute": 120, "burst": 30 }
}
\end{verbatim}
  \end{minipage}
  \caption{Listado: configuración JSON de ejemplo}
  \label{lst:json}
\end{figure}

\section{Ecuaciones}
A continuación se muestran ejemplos de ecuaciones en línea y desplegadas. Véase la ecuación \ref{eq:pitagoras} y el sistema en \ref{eq:sistema}.

Texto con matemática en línea: si $x, y \in \mathbb{R}$, entonces $z = x + y$.

Ecuación numerada:
\begin{equation}
  a^2 + b^2 = c^2
  \label{eq:pitagoras}
\end{equation}

Derivación en varias líneas con alineación:
\begin{eqnarray}
  S &=& \sum_{i=1}^{n} i \\
  &=& 1 + 2 + \cdots + n \\
  &=& \frac{n(n+1)}{2} \label{eq:suma}
\end{eqnarray}

Matriz con notación de corchetes:
\[
  A = \left[
    \begin{array}{ccc}
      1 & 0 & 2 \\
      -1 & 3 & 1
    \end{array}
  \right]
\]

Función a trozos:
\[
  f(x) = \left\{
    \begin{array}{ll}
      x^2 & \mbox{si } x \ge 0, \\
      -x  & \mbox{si } x < 0.
    \end{array}
    \right.
    \label{eq:sistema}
  \]

  Producto escalar y norma:
  \[
    \langle \mathbf{u}, \mathbf{v} \rangle = \sum_{i=1}^{n} u_i v_i,
    \qquad
    \|\mathbf{u}\| = \sqrt{\langle \mathbf{u}, \mathbf{u} \rangle }.
  \]

  Referencias cruzadas de ejemplo: figura \ref{fig:placeholder2}, tabla \ref{tab:agrupada}, listado \ref{lst:python} y ecuación \ref{eq:suma}.
