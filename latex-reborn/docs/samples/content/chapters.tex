% Sample chapters: mix of very long and short titles, with and without content.

% Short title only, with content
\chapter{Introducción}
Este capítulo introduce el contexto y los objetivos del trabajo. Presenta el problema, la motivación y el alcance general del proyecto.

% Long title with an explicit short title for ToC/headers, with content
\chapter[Marco Teórico]{Marco Teórico Extenso: Un recorrido exhaustivo por conceptos, definiciones, antecedentes, taxonomías y fundamentos necesarios para comprender la propuesta presentada en este Trabajo Final de Grado}
En este capítulo se desarrollan los conceptos clave, se revisa la literatura relevante y se establecen las bases teóricas que sustentan la solución propuesta.

% Short title, intentionally without content
\chapter{Trabajo Relacionado}
% (Intencionalmente sin contenido)

% Very long title with short variant, with content
\chapter[Metodología]{Metodología: Diseño, planificación, ejecución de experimentos, recolección y análisis de datos, y consideraciones éticas y prácticas en entornos reales y simulados con múltiples iteraciones y validaciones}
Se describe el enfoque metodológico, los pasos seguidos y las herramientas empleadas. Asimismo, se detallan los criterios de evaluación y los procedimientos de validación.

% Short title, minimal content
\chapter{Resultados}
Se presentan los resultados principales, incluyendo métricas, observaciones y hallazgos relevantes derivados de los experimentos realizados.

% Very long title, with content
\chapter[Conclusiones]{Conclusiones y Trabajo Futuro con un título extraordinariamente largo que estresa el diseño del índice, cabeceras y marcadores, verificando el manejo correcto de textos extensos sin desbordes ni truncamientos indeseados}
Se resumen los aportes del trabajo, se discuten las limitaciones y se proponen líneas de trabajo futuro.

% Short title, intentionally without content
\chapter{Agradecimientos}
% (Intencionalmente sin contenido)

% Unnumbered chapter with very long title (optional)
\chapter*{Glosario Ampliado y Lista de Acrónimos con una Denominación Extensa para Probar Encabezados y Tabla de Contenidos en Diferentes Anchos de Página}
% Para añadirlo al índice, descomente la siguiente línea:
% \addcontentsline{toc}{chapter}{Glosario Ampliado y Lista de Acrónimos}
Este apartado reúne términos clave y sus definiciones, así como los acrónimos utilizados a lo largo del documento.
