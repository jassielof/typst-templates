\documentclass[titlepage,a4paper,12pt]{article}

% Preamble: simple macros and settings without external packages
\title{Comprehensive Article Class Sample}
\author{First Author\thanks{Supported by XYZ Foundation} \and Second Author}
\date{\today}

\newcommand{\code}[1]{\texttt{#1}} % simple macro for inline code

\begin{document}
\pagenumbering{roman}
\maketitle

\begin{abstract}
  This abstract summarizes the purpose, methods, results, and conclusions in a few sentences.
\end{abstract}

\tableofcontents
\listoffigures
\listoftables
\clearpage
\pagenumbering{arabic}

\section{Introduction}
This sample showcases common \LaTeX{} article features for porting to Typst. It references Figure~\ref{fig:placeholder}, Table~\ref{tab:sample}, and Equation~\eqref{eq:pythagoras}. There is a footnote\footnote{A standard footnote.} and a margin note.\marginpar{Margin note.}

Here is a footnote mark placed now\footnotemark{} with its text defined later in the page.
\footnotetext{Footnote text paired with the earlier mark.}

\subsection{Text Emphasis and Fonts}
Normal, \emph{emphasized}, \textbf{bold}, \textit{italic}, \texttt{monospace}, \textsf{sans}, \textsc{small caps}.
Family switches: {\rmfamily roman}, {\sffamily sans}, {\ttfamily typewriter}.
Size switches: {\tiny tiny}, {\scriptsize scriptsize}, {\footnotesize footnotesize}, {\small small}, {\normalsize normalsize}, {\large large}, {\Large Large}, {\LARGE LARGE}, {\huge huge}, {\Huge Huge}.

\subsubsection{Paragraphing}
This is one paragraph. \par
This is another paragraph started with \verb|\par|. A blank line also starts a new paragraph.

\paragraph{Run-in Paragraph}
Run-in paragraph text continues inline with its heading to demonstrate styling.

\subparagraph{Subparagraph}
Subparagraph text demonstrates the smallest heading level.

\section{Alignment and Spacing}
\begin{center}
  Centered line using the center environment.
\end{center}

\begin{flushleft}
  Left-aligned line using the flushleft environment.
\end{flushleft}

\begin{flushright}
  Right-aligned line using the flushright environment.
\end{flushright}

No indent: \noindent This line is not indented.

Horizontal space: A\hspace{1em}B\hfill C. Nonbreaking space example: Figure~\ref{fig:placeholder}.

Vertical space:
First line.\vspace{6pt}

Second line.

\section{Lists}
\subsection{Itemize}
\begin{itemize}
  \item First item
  \item Second item with a nested list
    \begin{itemize}
      \item Nested item
    \end{itemize}
\end{itemize}

\subsection{Enumerate}
\begin{enumerate}
  \item First
  \item Second
  \item Third
\end{enumerate}

\subsection{Description}
\begin{description}
  \item[Term] Definition text goes here.
  \item[Another term] Another definition.
\end{description}

\section{Math}
Inline math: $a^2 + b^2 = c^2$.

Displayed equation:
\begin{equation}
  \label{eq:pythagoras}
  a^2 + b^2 = c^2
\end{equation}

Multi-line with alignment using \texttt{eqnarray}:
\begin{eqnarray}
  f(x) & = & x^2 + 2x + 1 \nonumber \\
  & = & (x+1)^2
\end{eqnarray}

Matrix with \texttt{array}:
\[
  \left(
    \begin{array}{cc}
      1 & 2 \\
      3 & 4
    \end{array}
  \right)
\]

\section{Floats}
\subsection{Figure}
\begin{figure}[htbp]
  \centering
  % Placeholder box: 3in wide, 2in tall
  \fbox{\rule{0pt}{2in}\rule{3in}{0pt}}
  \caption{Placeholder figure}
  \label{fig:placeholder}
\end{figure}

\subsection{Table}
\begin{table}[htbp]
  \centering
  \caption{Sample table}
  \label{tab:sample}
  \begin{tabular}{lcr}
    \hline
    Left & Center & Right \\
    \hline
    A & B & C \\
    1 & 2 & 3 \\
    \hline
  \end{tabular}
\end{table}

\subsection{Minipages}
\begin{figure}[htbp]
  \centering
  \begin{minipage}[t]{0.45\linewidth}
    \centering
    \fbox{\rule{0pt}{1.5in}\rule{2in}{0pt}}\\
    Small box A
  \end{minipage}
  \hfill
  \begin{minipage}[t]{0.45\linewidth}
    \centering
    \fbox{\rule{0pt}{1.5in}\rule{2in}{0pt}}\\
    Small box B
  \end{minipage}
  \caption{Two minipages side by side}
  \label{fig:minipages}
\end{figure}

\section{Verbatim and Code}
Inline \verb|for (i=0; i<10; ++i) {}| and a block:
  \begin{verbatim}
  function greet(name) {
    return "Hello, " + name;
  }
  \end{verbatim}

Custom macro for code: \code{printf("hello");}

\section{Quotations and Verse}
\subsection{Quote}
\begin{quote}
  A short quotation with one or more paragraphs.
\end{quote}

\subsection{Quotation}
\begin{quotation}
  A longer quotation where paragraphs are indented to indicate extended quotes.
\end{quotation}

\subsection{Verse}
\begin{verse}
  Roses are red,\\
  Violets are blue.\\
  Poetic lines break with \verb|\\|.
\end{verse}

\section{Cross-references}
See Section~\ref{sec:methods} on page~\pageref{sec:methods}. Also see Figure~\ref{fig:minipages}.

\section{Methods}
\label{sec:methods}
This section is referenced earlier.

\section{Boxes and Rules}
\fbox{Framed text} \quad \framebox[3cm][c]{3cm wide}
\par
\mbox{Unbreakable text segment} \par
\makebox[5cm][r]{Right-justified in 5cm}
\par
Rules: \rule{1cm}{0.4pt} thin line, \rule{2mm}{5mm} rectangle.

\section{Page Style}
\thispagestyle{plain}
\pagestyle{plain}

\section{Counters and Numbering}
This is Section \thesection.
\setcounter{secnumdepth}{5}
\setcounter{tocdepth}{3}
Numbered subparagraphs are enabled below.
\subsubsection{Numbering Depth Demo}
\paragraph{A Paragraph}
\subparagraph{A Subparagraph}

\section*{Unnumbered Section}
This section is unnumbered but added to the TOC manually.
\addcontentsline{toc}{section}{Unnumbered Section in TOC}

\appendix
\section{First Appendix}
Appendix sections are lettered. See Appendix~\ref{sec:appB}.

\section{Second Appendix}
\label{sec:appB}
Second appendix content.

\begin{thebibliography}{9}
  \bibitem{lamport94}
  Leslie Lamport,
  \emph{\LaTeX: A Document Preparation System},
  Addison--Wesley, 2nd edition, 1994.
\end{thebibliography}

\end{document}
